%!TEX root = ../schoedon.tex

%%%%%%%%%%%%%%%%%%%%%%%%%%%%%%%%%%%%%%%%%%%%%%%%%%%%%%%%%%%%%%%%%%%%%%%%%%%%%%%%
\cleardoublepage              %%% INTRODUCTION                               %%%
\chapter{Introduction}
  \label{chap:intro}
  This thesis proposes a new technique of web-based transportation network
  visualization with the application of network-based accessibility maps.\par

  \section{Motivation}
    \label{sec:intro:motiv}
    Different approaches for rendering maps in web applications exist.
    Widely used architectures are raster or vector services that provide web
    mapping applications with geographic data in client/server models.
    Raster data is efficiently mapped and rendered server-side using a
    static, predefined display. Vector data can be rendered client-side
    using dynamic mappings and interactive stylization.\par

    \begin{figure}[h]
      \centering
      \includegraphics[width=\linewidth]
        {./img/screenshot-teaser-7200s-publictransport.png}
      \caption{A network-based accessibility map visualization.}
      \label{fig:intro:teasr}
    \end{figure}

    However, these techniques do not supply a convenient approach to map and
    render high detailed, unfiltered transportation network geometries
    required for applications such as accessibility analytics (compare Figure
    \ref{fig:intro:teasr}).\par

  \section{Definitions}
    \label{sec:intro:def}
    Both the terms \textit{transportation networks} and \textit{accessibility
    maps} are ambitious and will be defined for future usage in this
    thesis.\par

    \begin{description}
      \item[Transportation Network] is a routable geo-data graph representation.
        It includes all path types from motorways to residential roads
        including footways and cycleways [??]. It does not include railways or bus
        guide-ways which are not accessible by private means of transport.
      \item[Accessibility Maps] are a tool of spatial mobility analytics. They
        allow single source shortest path (\acrshort{sssp}) display of
        geographic areas [??]. Accessibility can used be as a synonym for
        \textit{reachability} in this case. It's not related to services or
        environments for people who experience disabilities.
    \end{description}

  \section{Problem statement}
    \label{sec:intro:probl}

    % existing approaches / drawbacks
      % creation requires generalization
      % lacks precision wrt network
      % does not scale for more detailed display

    Rendering transportation networks in web-based applications remains a
    performance critical task due to the complexity of the underlying geo-data.
    An OpenStreetMap (\acrshort{osm}) graph dataset of the Berlin/Brandenburg
    region as
    of June 2016 contains around $2.4 \cdot 10^6$ edges spanned by
    $2.9 \cdot 10^6$ vertices~\cite{STHD2016}. Using
    classic approaches to render such a dataset in a web browser either
    leaves users with a predefined, static mapping (raster data) or a notable
    computation-intense rendering process (vector data).\par

    The two aforementioned techniques of mapping and rendering web maps are
    widely established and have proven effective. But both approaches have
    certain drawbacks.\par

%   Real-time rendering transportation networks as scenery for data visualization in web-based applications is a performance critical task depending on the geometric complexity of the network and associated travel times. For example, an OSM dataset of the Berlin region comprises approx. $9 \cdot 10^5$ edges (Oct.~2015). Using traditional visualization approaches using web browsers either faces users with a predefined, static filtering and mapping (raster data) or a notable computation-intense rendering process (vector data). These two fundamental approaches covering filtering, mapping, and rendering web-based maps are widely established and have proven to be effective, but exhibit drawbacks:

    Data transmitted in pre-rendered raster data formats (e.g., \acrshort{png},
    \acrshort{jpg}) does
    not require any client-side processing prior to rendering, can be compressed
    as well as cached~\cite{ESRI2006}, and is used by major web-mapping services
    such as Google or Bing maps [??]. However, one major disadvantage is the
    lack for client-side filtering and mapping without requesting a complete map
    tile reload. Web services using this technology approach this by using small
    vector overlays displaying additional user-styled information [??], but it
    is not possible to access the the map data itself.\par

    In contrast thereto, geo-data transmitted using vector formats
    (e.g., \acrshort{json}, \acrshort{gml}) enable
    client-side filtering, mapping, and rendering. This client-side processing
    however introduces a major performance impact: both the data processing and
    rendering are usually performed on central processing units (\acrshort{cpu})
    using JavaScript (\acrshort{js}) algorithms. Recent approaches do support
    hardware-accelerated rendering (\acrshort{gpu}) but lack functionality for
    client-side vector data processing~\cite{Gaffuri2012}.\par

%   Thus, both approach changes in filtering (e.g., selecting a travel time threshold) or mapping (e.g., color mapping, line styles etc.) would result in a complete data re-transmission, loading, and processing. To summarize, an interactive visualization technique for web-based accessibility maps should adhere to the following requirements and challenges: it should support a web-based, hardware-accelerated implementation using WebGL~\cite{Parisi2012}~(R1); a standardized and compact data representation that allows for decoupling network geometry from temporal data to reduce data transmission and updates~(R2); as well as enable interactive client-side filtering, mapping, and rendering for visual feedback~(R3).

    Therefore, high detailed network analysis and visualization is often
    performed using Geographic Information Systems (\acrshort{gis}). Such
    systems
    exploit the computation power of desktop workstations but posses limited
    applicability to everyday life, due to data availability and access, as well
    as expert domain knowledge required by a user.\par

  \section{Contributions}
    \label{sec:intro:contr}

    In the light of powerful, dedicated graphics hardware being not only
    available on personal computers (\acrshort{pc}) but also
    on small, mobile devices nowadays, this thesis suggests new techniques for
    client-side rendering of web-maps with complex geometries on graphic
    processing units (\acrshort{gpu}).\par

    The challenge of this work is twofold interesting. On the one hand it is
    important to enable rendering using GPU-based techniques like
    the web graphics library (\acrshort{webgl}) \cite{Jackson2016}.
    This allows dynamic, interactive and user-defined styles to be rendered
    directly on the client's device. But on the other hand it is a must to
    completely eliminate the client-side mapping of the geo-data as this becomes
    a major performance bottleneck with increasing data complexity.\par

    This thesis proposes a new approach for web-based visualization of
    accessibility maps based on transportation network data. This geometry-based
    approach uses vector data (lines) stored and transmitted using the new
    standardized graphics library transmission format (\acrshort{gltf}) [??],
    which reduces performance-critical
    computations in the visualization client (i.e., coordinate transformations)
    and thus facilitates real-time rendering with high run-time performance
    yielding low client response times. To summarize, this work makes the
    following contributions:\par

    \begin{enumerate}[\label=({C}1)]
      \item It presents a concept to decouple the visualization geometry and
        data items for interactive web-based accessibility maps based on
        \acrshort{webgl}. %@TODO major C missing
      \item It demonstrates the effectiveness of this approach by a
        comparative performance evaluation of different implementation
        variants.
    \end{enumerate}

    The technique presented in this thesis is a graphics library (\acrshort{gl})
    based approach rather than classic vector- or raster-based provisions.
    It shows various advantages:\par

    \begin{enumerate}[\label=({A}1)]
      \item Its usage is not limited to stationary desktop systems but available
        on a variety of devices (esp. mobile).
      \item Potentially massive data sources are not required to be completely
        transmitted, stored, or managed.
      \item Implementations based on web-services and \acrshort{webgl} can be
        easily integrated into existing systems and visualization frameworks.
    \end{enumerate}

    In contrast to existing accessibility-map visualization concepts, this work
    focuses on visualizing the travel time data directly on the respective
    transportation network features, rather than (possibility generalized)
    polygons~\cite{Glander2010} or specific graph layouts~\cite{Krause2012}.
    This enables a precise mapping of travel data to the geo-referenced
    transportation network. However, considering the high geometric complexity
    (vertices, primitives) introduced by increasing quality of transportation
    networks~\cite{Zielstra2010}, e.g., of massive open data transportation
    networks (OpenStreetMap (\acrshort{osm}) or General Transit Feed
    Specification (\acrshort{gtfs})),
    an implementation of an interactive web-based visualization technique
    comprises a number of conceptual and technical challenges.\par

    It maintains the goal to allow real-time rendering with outstanding
    performance and very low response times for the client.\par

