%!TEX root = ../schoedon.tex

\cleardoublepage
\chapter{Overview and related work}
  \label{chap:overv}

  Table \ref{tab:overv:relat} (page \pageref{tab:overv:relat}) shows a first
  overview of related graphics, applications and publications in the broad
  field of mobility analytics and accessibility mapping techniques.\par

  \begin{table}[htp]
    \tiny \centering
    \begin{tabular}{r|l|l|l}
     \textbf{Year} & \textbf{Authors} & \textbf{Contribution} & \textbf{Type} \\
     \hline
      1881 & Galton \cite{galton1881construction} & Construction of isochronic passage-charts  & Publication  \\
      1959 & Hansen \cite{hansen1959accessibility} & How accessibility shapes land use & Publication \\
      1972 & Armstrong \cite{armstrong1972network} & Network analysis of airport accessibility  & Publication  \\
      1978 & Muller \cite{muller1978mapping} & Mapping of travel times  & Publication  \\
      1981 & Sugiura \cite{Sugiura1981} & Japan travel time maps  & Graphic  \\
      1986 & Hall \cite{hall1986fastest} & Network with random time-dependent travel times  &  Publication \\
      1989 & Cauvin et al. \cite{cauvin1989cartographic} & The piezopleth maps method  & Publication  \\
      1993 & Spierkermann et al. \cite{spiekermann1993zeitkarten} & Time maps for spatial planning  & Publication  \\
      1994 & Spierkermann et al. \cite{spiekermann1994new} & Time-space maps of Europe  & Publication  \\
      1996 & Gutierrez et al. \cite{gutierrez1996accessibility} & Accessibility analysis of the European road network  & Publication  \\
      1998 & Fritz et al. \cite{fritz1998accessibility} &  Accessibility as wilderness indicator  & Publication  \\
      1999 & Spiekermann \cite{spiekermann1999visualisierung} & Visualization of railway travel times & Publication \\
      2000 & Miller et al. \cite{miller2000gis} & GIS for measuring space-time accessibility in transportation & Publication \\
      2000 & O'Sullivan et al. \cite{o2000using} & Using desktop GIS for accessibility analysis in public transport  & Publication  \\
      2001 & Ran \cite{ran2001method} &  Method of providing travel time  & Patent \\
      2001 & Handy \cite{handy2002accessibility} & Accessibility versus mobility & Publication \\
      2002 & Lovett et al. \cite{lovett2002car} & Accessibility of general practitioner services  & Publication  \\
      2003 & Dailey et al. \cite{dailey2003design} &  Multi-modal transit management system & Publication  \\
      2005 & Auxhausen \cite{axhausen2005zeitkarten} & Time-maps of Switzerland  & Publication  \\
      2005 & Chronomap \cite{Chronomap} & Drive-time analysis  & Application  \\
      2005 & Karlin \cite{Karlin2005}  & London subway travel times  & Graphic \\
      2005 & McLaren \cite{McLaren2005} & Time travel with the London tube map  & Graphic  \\
      2005 & Travel Time Tube Map \cite{Carden2006} & Interactive travel time tube map  & Application  \\
      2006 & Lightfoot \cite{Lightfoot2006} & Use-cases for travel time maps  &  Publication  \\
      2006 & Pfoser et al. \cite{pfoser2006dynamic} &  Dynamic travel time maps & Publication  \\
      2006 & Rüegg \cite{Ruegg2006} & Impact of public transport on travel times  & Poster \\
      2006 & Street \cite{street2006timecontours} & Isochrone visualization to describe transport network costs  & Master's Thesis  \\
      2007 & Irving \cite{Irving2007} & Use-cases for travel time maps  & Publication  \\
      2007 & Nies et al. \cite{neis2007webbasierte} & Web-based accessibility analysis  & Publication  \\
      2008 & Bauer et al. \cite{bauer2008computing} & Isochrones in multi-modal transportation networks  & Publication  \\
      2008 & Mapumental \cite{Mapumental}  &  Maps that show time & Application  \\
      2008 & Uchida et al. \cite{Uchida2008} & Travel time to major cities  & Graphic  \\
      2009 & FreeMapTools \cite{Freemaptools} & How far can I travel  & Application  \\
      2009 & Uchida et al. \cite{uchida2009agglomeration} & Measure of urban concentration  & Publication  \\
      2010 & Antrim et al. \cite{antrim2013many} & Use-cases for GTFS data & Publication  \\
      2010 & Campenhout \cite{van2010travel} & Travel time maps  & Master's Thesis  \\
      2010 & Glander et al. \cite{Glander2010} & Accessibility maps to visualize quality of mobility  &  Publication \\
      2010 & Lei et al. \cite{lei2010mapping} & Mapping transit-based access  & Publication  \\
      2010 & Marciuska et al. \cite{marciuska2010determining} & determining objects within isochrones  & Publication  \\
      2010 & Müller et al. \cite{Mueller2010} & Distance transformations for accessibility mapping  &  Publication \\
      2010 & Time Maps \cite{TimeMaps} & Public transport time maps of the Netherlands & Application  \\
      2011 & Birchler \cite{birchler2011computing} & Isochrones in multi-modal public transport networks  & Bachelor's Thesis  \\
      2011 & Gemmel \cite{gemmel2012hedonic} &  Effects of walk-ability and public transit & Master's Thesis  \\
      2011 & Gamper et al. \cite{gamper2011defining} & Defining isochrones in multi-modal spatial Networks & Publication  \\
      2011 & Isochrone.ch \cite{IsochroneCh}  & Isochrones in schedule-based public transport  & Application  \\
      2011 & Li et al. \cite{li2011dynamic} & Dynamic accessibility mapping  & Publication  \\
      2011 & Söderström \cite{soderstrom2011personal} & Internet-driven maps based on time distances  & Publication  \\
      2012 & Byrd \cite{Byrd2012} & Visualizing urban accessibility  & Publication  \\
      2012 & Gamper et al. \cite{gamper2012scalable} & Computation of isochrones with network expiration  & Publication  \\
      2012 & Conveyal \cite{Conveyal} & Transportation - land use analysis  & Application  \\
      2012 & Hollburg et al. \cite{hollburghier} & Interactive accessibility analysis in Potsdam & Publication  \\
      2012 & Mapnificent \cite{Mapnificent}  & Web-based reachability visualization of public transport  & Application  \\
      2012 & Mertens \cite{meertens2012} & Travel Time Maps of urban areas in the Netherlands  & Graphic  \\
      2012 & TripTropNYC \cite{TriptropNYC} & Web-based accessibility visualization in New York  & Application  \\
      2013 & Innerebner \cite{Innerebner2013} & Isochrones in multi-modal spatial networks  & PhD Thesis  \\
      2013 & Innerebner et al. \cite{innerebner2013isoga} & Web-based geospatial reachability analysis tool  & Publication  \\
      2013 & Transit Time NYC \cite{TransitTimeNYC} & Web-based subway transit times in New York & Application \\
      2013 & Tran et al. \cite{tran2013go_sync} & Synchronizing transit data between GTFS and OSM  & Publication  \\
      2014 & Gortana et al. \cite{gortanaisoscope} & Visualizing temporal mobility variance  & Publication  \\
      2014 & Hollburg \cite{Hollburg2014} & Interactive analysis and visualization of accessibility & Master's Thesis  \\
      2014 & Isoscope \cite{Isoscope} & Visualizing mobility with isochrone maps  & Application  \\
      2014 & Route360° \cite{Route360} & Web-based travel time analysis  & Application  \\
      2014 & Krismer et al. \cite{krismer2014incremental} & Incremental calculation of isochrones  & Publication  \\
      2014 & Voll \cite{vollerreichbarkeiten} & Accessibility analysis of the Alps  & Publication  \\
      2015 & TravelTimePlatform \cite{TravelTimePlatform} & Search and filter by time not distance  & Application  \\
      2015 & Yin et al. \cite{yin2015understanding} & Web-based accessibility analysis and travel time displays  & Publication  \\
      2016 & Schoedon et al. \cite{STHD2016} & Web-based visualization of transportation networks  & Publication  \\
    \end{tabular}
    \caption{Selected related work.}
    \label{tab:overv:relat}
  \end{table}

  \section{Noteworthy publications and theses}
    \label{sec:overv:publc}

    % pubs %%%%%%%%%%%%%%%%%%%%%%%%%%%%%%%%%%%%%%%%%%%%%%%%%%%%%%%%%%%%%%%%%%%%%

    % galton first isochronic passage chart 1881
    % distances that can be traversed in equal times
    % offered merely as generalization
    % primary object offering new principle
    % isochronic maps easily constructed for continental travel or home excursions
    %%% GRAPHIC ISO PSG CHART

%    Back in 1881, Francis Galton created the Isochronic Passage Chart for the
%    Royal Geographic Society (figure \ref{fig:overv:isopc}). It depicts the world
%    as a thematic map with an overlay of colour-coded isochrones to show the
%    distances that can be traveled in equal times. It uses the isochronic
%    visualization as a rough generalization and suggests future areas of
%    application might be \enquote{continental travel} or \enquote{home
%    excursions} \cite{galton1881construction}.\par

    \begin{figure}[t]
      \subcaptionbox{
        \label{fig:overv:isopc}
        The Isochronic Passage Chart by Francis Galton, 1881
        \cite{galton1881construction}.}
      {\includegraphics[width=0.32\textwidth]{./img/overv-isopc.jpg}}
      \hfill
      \subcaptionbox{
        \label{fig:overv:nodac}
        Analysis of Airport Accessibility in South Hampshire, 1972
        \cite{armstrong1972network}.}
      {\includegraphics[width=0.32\textwidth]{./img/overv-nodac.png}}
      \hfill
      \subcaptionbox{
        \label{fig:overv:maptt}
        The Mapping of Travel Time in Edmonton, Alberta, 1978
        \cite{muller1978mapping}.}
      {\includegraphics[width=0.32\textwidth]{./img/overv-maptt.png}}
      \subcaptionbox{
        \label{fig:overv:patnt}
        Method of Providing Travel Time, 2001
        \cite{ran2001method}.}
      {\includegraphics[width=0.32\textwidth]{./img/overv-patnt.png}}
      \hfill
      \subcaptionbox{
        \label{fig:overv:berln}
        Visualization of Quality of Mobility in Public Transport, 2010
        \cite{Glander2010}.}
      {\includegraphics[width=0.32\textwidth]{./img/overv-berln.png}}
      \hfill
      \subcaptionbox{
        \label{fig:overv:potsd}
        Network-Based Accessibility Visualization in Potsdam, 2012
        \cite{hollburghier,Hollburg2014}.}
      {\includegraphics[width=0.32\textwidth]{./img/overv-potsd.png}}
      \caption{Foo bar.}
      \label{fig:overv}
    \end{figure}

    In 1881 Francis Galton constructed the first isochronic map for the Royal
    Geographic Society in London (figure \ref{fig:overv:isopc}).

    % hansen 1959
    % accessibility indicator urban transportation systems
    % serving residents in urban areas
    % method for determining accessibility patterns within metropolitan areas
    % defines accessibility is potential of opportunities for interaction

    % armstrong 1972
    % network analysis accessibility
    % transport costs play an important role
    % here: economic evaluation of airports
    % road travel times and weighting by importance for the application and population
    % among first computer based accessibility analysis: C.D.C. 6000 computer
    % road network matrix, resulting graph 156 nodes
    %%% GRAPHIC FIG 3or4

    % muller 1978
    % mapping of travel time
    % time distance: time required to travel specific distance
    % derived space is non-euclidean not mappable in two dimensions
    % provides a numerical approach
    %%% GRAPHIC Fig 1

    % ran Google patent travel time 2001
    % patent, method of providing travel time
    % personalized multi-modal travel prediction and trip decision support system
    % traffic forecast maps: travel speed maps, travel time maps, travel cost maps, etc.
    %%% GRAPHIC fig 7a origin based predictive travel time (vs. destination based)
    % contour line represents predictive minimal travel time from the origin

    % street thesis application isochrones as transportation network costs display 2006
    % time contours java application: isochrone generating system (contours, isolines)
    % time based maps describe time accessibility of networks
    % isochrone implementations among most successful, does not disrupt underlying map
    % contours are method to display 3rd dimension on two (like height, isohypses)
    %%% same works for time: isochrones
    %%% FIG: 56 or 57 global integration of london (axial map) of network
    % most of time renders isochrones

    % neis et al. 2007
    % web based accessibility analysis, suggests accessibility analysis service (AAS)
    % data provided in xml, web service in java (applet)
    % utilizes attributed street network graph
    % three methods of visualizing accessibility: isochrones, buffer, convex hull
    % creates elevation model where time is the 3rd dimension
    % calculates isohypses whcih are isochrones technical
    %%% FIG 13 accessibility maps buffer isolines convexhull

    % van campenhout summary overview travel time maps 2010
    % well organized summary: accessibility index, isochrone maps, anamorphosis maps

    % glander 2010
    % accessibility maps for the visualization of quality of mobility in public transport
    % calculates isochrones from weighted voronoi centers (?) and distance fields
    %%% FIG 1 public transport berlin

    % müller 2010
    % distance transformations for accessibility mapping in public transport domain
    % for concurrent web based accessibility mapping
    % surface based (euclidean) vs network based (non euclidean) distance computation
    % first who highlights: display of network is no only effective for the user but efficient and correct calc'able
    % generalized polygons introduce visual errors and high rendering demand

    % hollburg 2012
    % web based accessibility analysis, tourism in potsdam
    % network based analysis and visualization
    % web based provision
    %%% FIG hollburg 2014

    % innereber et al. isogar (2013)
    % figure (web app)

    % gortana 2014
    % isoscope web app
    % unified isochrone maps with time varying  travel data
    % instead multiple isolines for for travel times: mult isolines for times of day
    % reveals time-dependent spatial travel variance
    % FIG 1 24 layered shapes for each hour of the day

    % hollburg 2014
    %%% FIG 18
    % network based method developed in hollburg 2012 was comprehensible, but not efficient for huge transportation networks
    % generalization of road segments, usage of isochrones to vis
    % faster transmission of polygons due to smaller data volume (svg)
    % SONA (simple online network analysis)

    % yin et al. 2015
    % web based system to visualize multi-modal accessibility
    % integrated platform for accessibility analysis tasks
    % accessibility term: combination of mobility and potential
    %%% mobility: ability of movement
    %%% potential: number/size of destination opportunities
    % web based providing easy to use interface for users
    % light weight, users will not need to purchase or install software
    % accessible to a wider range of audiences
    % FIG 4accessibility view (choropleth), travel time view (isochrone)

    % this work in line with / based on hollburg 2012 and 2014 and the services motion intelligence offers (r360)

    % maps %%%%%%%%%%%%%%%%%%%%%%%%%%%%%%%%%%%%%%%%%%%%%%%%%%%%%%%%%%%%%%%%%%%%%

    % apps %%%%%%%%%%%%%%%%%%%%%%%%%%%%%%%%%%%%%%%%%%%%%%%%%%%%%%%%%%%%%%%%%%%%%


%    Google issued a patent in the United States of America on the
%    \enquote{Method of Providing Travel Time} (compare figure
%    \ref{fig:overv:patnt}).\par
%
%    \begin{figure}[htb]
%      \centering
%      \includegraphics[width=\linewidth]
%        {./img/overv-patnt.png}
%      \caption{The Method of Providing Travel Time by Bin Ran, 2000 \cite{ran2001method}.}
%      \label{fig:overv:patnt}
%    \end{figure}

  \section{Related applications and maps}
    \label{sec:overv:applc}

    % isoga, simplefleet, public transit travel time, simple online network analysis(hollburg 2014 FIG 7)

%    Time Travel (Oskar Karlin) 2005
%
%http://www.oskarlin.com/2005/11/29/time-travel/
%
%Oskar Karlin. Time travel, 2005. URL http://www.oskarlin.com/2005/11/29/time-travel/.
%Letzter Zugriff: 31.1.2010.
%
%Time Travel, Oskar Karlin, http://www.oskarlin.com/2005/11/29/time-travel/
%[Accessed 04 Jan 2006]
%
%#######################
%http://www.openrouteservice.org/
%Accessibility Analysis
%
%#######################
%https://github.com/mapbox/osrm-isochrone
%
%#######################
%WalkScore Travel Time API
%https://www.walkscore.com/professional/travel-time-api.php
%
%#######################
%OneBayArea, San Fransisco Plan 2040
%http://maps.planbayarea.org/travel_housing/
%
%#######################
%Graphhopper Direction API
%Isochrone API
%https://graphhopper.com/api/1/docs/isochrone/
%
%#######################
%Hollburg et al (2014)
%https://www.route360.net/
%
%#######################
%http://cartoo.dyndns.org/


% https://www.khronos.org/news/press/significant-gltf-momentum-for-efficient-transmission-of-3d-scenes-models

%%%%%%%%\chapter{Overview and related work}
%%%%%%%%  \label{chap:overv}

  %  Related work comprises basically accessibility map visualization, web-based
  %  visualization frameworks, and the rendering of transportation networks using
  %  GPUs. Glander et al. present an accessibility map visualization technique
  %  with a focus on polygon-based approaches~\cite{Glander2010}, and raster-based
  %  distance transforms~\cite{Mueller2010} which both lack precision in display
  %  offered by our approach. In~\cite{Yin2015}, a web-based system for visualization
  %  of multi-modal accessibility for multiple
  %  land-uses is presented. However, the visualization technique does not focus on the
  %  specifics of transportation network representations. Altmaier et al. (2003) was
  %  among the first to outline issues in web-based geovisualization
  %  applications~\cite{Altmaier2003}. In~\cite{Brabec2007}, challenges, requirements,
  %  and concepts of client-based browsing of spatial data on the World Wide Web are
  %  discussed. The presented system is based on a Java-Applet and does not exploit
  %  modern web technologies for rendering complex spatial data. Vaaraniemi et al. (2011)
  %  as well as Trapp et al. (2015) develop and evaluate approaches for transport network
  %  visualization utilizing modern graphics hardware \cite{Vaaraniemi2011,Trapp2015}.
  %  However, the presented rendering techniques can currently not be implemented
  %  using technologies for browser-based rendering and do no cover specifics of
  %  data representation and formats.\par

  % one way of communicating mobility information such as travel times are two dimensional maps
  % accessibility and mobility becomes a central topic in a growing number of fields
  % there are strong demands for corresponding web mapping components
  % based on massive geodata sets, such as osm

%%%%%%%%  \section{Accessibility analytics and mapping techniques}
%%%%%%%%    \label{sec:overv:accss}

%    A general motivation on web-based accessibility map applications in the context of geospatial analytics offers Hollburg et al. (2012) \cite{hollburghier}.\par

%    Glander et al. (2010) researches on visualization of accessibility maps with focus on polygon-based approaches \cite{Glander2010}.\par

    % isochronic passage chart (1881)
    % shortest path problem classes
    % Single pair shortest path (A to B)
    % Single source shortest path (A to N)
    % All pairs shortest path (N to M)

%%%%%%%%  \section{Geographic visualizations of transportation networks}
%%%%%%%%    \label{sec:overv:geovs}

    %Vaaraniemi et al. (2011) as well as Trapp et al. (2015) develop and evaluate solutions for transport network visualization utilizing modern computer graphics \cite{Vaaraniemi2011}\cite{Trapp2015}.\par

%%%%%%%%  \section{Web-based mapping components and services}
%%%%%%%%    \label{sec:overv:webmp}

    %Altmaier et al. (2003) was among the first to outline issues in web-based geovisualization applications \cite{Altmaier2003}. Klimke et al. (2011) proposed a camera path specification for geovisualization services on the web and mobile devices \cite{klimke2013service}.\par

%%%%%%%%  \section{Web-based rendering technologies}
%%%%%%%%    \label{sec:overv:webrn}

    %Both Coughlin (2013) and Trevett (2013) deliver outstanding motivations on why we need a standardized data format close to hardware devices for 3D applications on the web \cite{Coughlin2014}\cite{Trevett2012}.\par

    % raster, java, webgl

%%%%%%%%%%%%%%%%%%%%%%%%%%%%%%%%%%%%%%%%%%%%%%%%%%%%%%%%%%%%%%%%%%%%%%%%%%%%%%%%
%%%%%%%%%%%%%%%%%%%%%%%%%%%%%%%%%%%%%%%%%%%%%%%%%%%%%%%%%%%%%%%%%%%%%%%%%%%%%%%%
%%%%%%%%%%%%%%%%%%%%%%%%%%%%%%%%%%%%%%%%%%%%%%%%%%%%%%%%%%%%%%%%%%%%%%%%%%%%%%%%
%%%%%%%%%%%%%%%%%%%%%%%%%%%%%%%%%%%%%%%%%%%%%%%%%%%%%%%%%%%%%%%%%%%%%%%%%%%%%%%%
%%%%%%%%%%%%%%%%%%%%%%%%%%%%%%%%%%%%%%%%%%%%%%%%%%%%%%%%%%%%%%%%%%%%%%%%%%%%%%%%

%  \section{Accessibility analytics and mapping techniques}
%    \label{sec:overv:accss}
%
%  \section{Geographic visualizations of transportation networks}
%    \label{sec:overv:geovs}
%
%  \section{Web-based mapping components and services}
%    \label{sec:overv:webmp}
%
%  \section{Web-based rendering technologies}
%    \label{sec:overv:webrn}
