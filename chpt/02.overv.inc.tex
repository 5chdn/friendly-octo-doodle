%!TEX root = ../schoedon.tex

%%%%%%%%%%%%%%%%%%%%%%%%%%%%%%%%%%%%%%%%%%%%%%%%%%%%%%%%%%%%%%%%%%%%%%%%%%%%%%%%
\cleardoublepage              %%% OVERVIEW                                   %%%
\chapter{Overview and related work}
  \label{chap:overv}
  Related work comprises basically accessibility map visualization, web-based
  visualization frameworks, and the rendering of transportation networks using
  GPUs. Glander et al. present an accessibility map visualization technique
  with a focus on polygon-based approaches~\cite{Glander2010}, and raster-based
  distance transforms~\cite{Mueller2010} which both lack precision in display
  offered by our approach. In~\cite{Yin2015}, a web-based system for visualization
  of multi-modal accessibility for multiple
  land-uses is presented. However, the visualization technique does not focus on the
  specifics of transportation network representations. Altmaier et al. (2003) was
  among the first to outline issues in web-based geovisualization
  applications~\cite{Altmaier2003}. In~\cite{Brabec2007}, challenges, requirements,
  and concepts of client-based browsing of spatial data on the World Wide Web are
  discussed. The presented system is based on a Java-Applet and does not exploit
  modern web technologies for rendering complex spatial data. Vaaraniemi et al. (2011)
  as well as Trapp et al. (2015) develop and evaluate approaches for transport network
  visualization utilizing modern graphics hardware \cite{Vaaraniemi2011,Trapp2015}.
  However, the presented rendering techniques can currently not be implemented
  using technologies for browser-based rendering and do no cover specifics of
  data representation and formats.\par
  % one way of communicating mobility information such as travel times are two dimensional maps
  % accessibility and mobility becomes a central topic in a growing number of fields
  % there are strong demands for corresponding web mapping components
  % based on massive geodata sets, such as osm
  \section{Accessibility analytics and mapping techniques}
    \label{sec:overv:accss}
    A general motivation on web-based accessibility map applications in the context of geospatial analytics offers Hollburg et al. (2012) \cite{hollburghier}.\par
    Glander et al. (2010) researches on visualization of accessibility maps with focus on polygon-based approaches \cite{Glander2010}.\par
    % isochronic passage chart (1881)
    % shortest path problem classes
    % Single pair shortest path (A to B)
    % Single source shortest path (A to N)
    % All pairs shortest path (N to M)
  \section{Geographic visualizations of transportation networks}
    \label{sec:overv:geovs}
    Vaaraniemi et al. (2011) as well as Trapp et al. (2015) develop and evaluate solutions for transport network visualization utilizing modern computer graphics \cite{Vaaraniemi2011}\cite{Trapp2015}.\par
  \section{Web-based mapping components and services}
    \label{sec:overv:webmp}
    Altmaier et al. (2003) was among the first to outline issues in web-based geovisualization applications \cite{Altmaier2003}. Klimke et al. (2011) proposed a camera path specification for geovisualization services on the web and mobile devices \cite{klimke2013service}.\par
  \section{Web-based rendering technologies}
    \label{sec:overv:webrn}
    Both Coughlin (2013) and Trevett (2013) deliver outstanding motivations on why we need a standardized data format close to hardware devices for 3D applications on the web \cite{Coughlin2014}\cite{Trevett2012}.\par
    % raster, java, webgl
%%%%%%%%%%%%%%%%%%%%%%%%%%%%%%%%%%%%%%%%%%%%%%%%%%%%%%%%%%%%%%%%%%%%%%%%%%%%%%%%
