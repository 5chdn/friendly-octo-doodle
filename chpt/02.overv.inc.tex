%!TEX root = ../schoedon.tex

\cleardoublepage
\chapter{Overview and related work}
  \label{chap:overv}

  \section{Accessibility analytics and mapping techniques}
    \label{sec:overv:accss}

    % Paper
    %%%%%%%%%%%%%%%%%%%%%%%%%%%%%%%%%%%%%%%%%%%%%%%%%
    % 1881 Galton et al.       ********
    % 1978 Muller
    % 1981 Sugiura
    % 1986 Hall
    % 1989 Cauvin et al.
    % 1993 Spierkermann et al.
    % 1994 Spierkermann et al.
    % 1996 Gutierrez et al.
    % 1998 Fritz et al.
    % 1999 Spiekermann
    % 2000 O'Sullivan et al.
    % 2001 Ran (Google Patent)
    % 2002 Lovett et al.
    % 2003 Dailey et al.
    % 2005 Auxhausen
    % 2005 Chronomap (Brochure)
    % 2005 Karlin
    % 2005 McLaren
    % 2005 Travel Time Tube Map (Project)
    % 2006 Cheng et al.
    % 2006 Kasper et al.
    % 2006 Lightfoot
    % 2006 Pfoster et al.
    % 2006 Rüegg
    % 2006 Street              ********
    % 2007 Innerebener et al.
    % 2007 Irving
    % 2007 Nies et al.
    % 2008 Bauer et al.
    % 2008 Cheng et al.
    % 2008 Mapumental (Project)
    % 2008 Uchida et al.
    % 2009 Uchida et al.
    % 2010 Antrim et al.
    % 2010 Campenhout          ********
    % 2010 Glander et al.      ********
    % 2010 Lei et al.
    % 2010 Marciuska et al.
    % 2010 Müller et al.       ********
    % 2011 Birchler
    % 2011 Gemmel
    % 2011 Gamper et al.
    % 2011 Isochrone.ch (Project)
    % 2011 Li et al.
    % 2011 Söderström
    % 2012 Byrd
    % 2012 Gamper et al.
    % 2012 Hollburg et al.     ********
    % 2012 Mapnificent (Project)
    % 2012 Mertens
    % 2012 TripTropNYC (Porject)
    % 2012 Time Maps (Project)
    % 2013 Innerebner
    % 2013 Innerebner et al.
    % 2013 Tran et al.
    % 2014 Gortana et al.      ********
    % 2014 Hollburg            ********
    % 2014 Isoscope (Project)
    % 2014 Krismer et al.
    % 2014 Voll
    % 2015 TravelTimePlatform (Project)
    % 2016 Schoedon et al.     ********

    % Projects
    %%%%%%%%%%%%%%%%%%%%%%%%%%%%%%%%%%%%%%%%%%%%%%%%%
    % 2005 Chronomap: MapInfo IT
    % Demo-Film: Erzeugung von Zeit-Isochronen mit ChronoMap für MapInfo Professional.
    % http://www.datagis.com/ChronoMap/ChronoMap.html
    % http://www.datagis.com/geodaten-geomarketing-geoconsulting.php?nid=30&kid=3
    %
    % Demo-ChronoMap-MapInfo-IT.avi
    % https://www.youtube.com/watch?v=QjRW55z7SHA
    %
    % ChronoMap 2005 Brochure
    %
    % http://www.theexmiles.com/programmingtools/chronomapdeveloperkit.html

    % Projects
    %%%%%%%%%%%%%%%%%%%%%%%%%%%%%%%%%%%%%%%%%%%%%%%%%
    % 2005 Carden: Travel Time Tube Map
    % T. Carden: Travel Time Tube Map (2005)
    % Tom Carden. Travel Time Tube Map. http://www.tom-carden.co.uk/p5/tube_map_
    % travel_times/applet/.
    % http://www.tom-carden.co.uk/2006/02/21/tube-travel-contours
    % http://www.tom-carden.co.uk/p5/tube_map_travel_times/applet/
    % Tom Carden. Tube Travel Contours, 2006. URL http://www.tom-
    % carden.co.uk/2006/02/21/tube-travel-contours/. Letzter Zugriff 31.1.2010.

    % Projects
    %%%%%%%%%%%%%%%%%%%%%%%%%%%%%%%%%%%%%%%%%%%%%%%%%
    % 2009 FreeMapTools: How far can I travel
    % https://www.freemaptools.com/how-far-can-i-travel.htm

    % Projects
    %%%%%%%%%%%%%%%%%%%%%%%%%%%%%%%%%%%%%%%%%%%%%%%%%
    % 2008 Irving, Lightfoot, et al.: Mapumental
    % [CITATION] Travel-time Maps and their Uses
    % C Lightfoot, T Steinberg - 2006 - … Society, funded by UK Department of …
    %
    % Travel-time Maps and their Uses, Chris Lightfoot and Tom Steinberg,
    % http://www.mysociety.org/2006/travel-time-maps/ [Accessed 10 June 06]
    %
    % @misc{lightfoot2006travel,
    %   title={Travel-time Maps and their Uses},
    %   author={Lightfoot, Chris and Steinberg, Tom},
    %   year={2006},
    %   publisher={My Society, funded by UK Department of Transport (www. mysociety. org/2006/travel-time-maps/index. php)}
    % }
    %
    % https://www.mysociety.org/2006/03/04/travel-time-maps-and-their-uses/
    % https://www.mysociety.org/2007/03/05/more-travel-time-maps-and-their-uses/
    % https://mapumental.com/ Lightfoot et al
    %
    % Parkes, Duncan; Somerville, Matthew; Broadbent, Jedidiah; Glushkova, Kristina; Lightfoot, Chris; Irving, Francis (2013): Mapumental Property. London: myScociety. Online verfügbar unter http://property.mapumental.com/, zuletzt geprüft am 27.12.2013.
    %
    % Chris Lightfoot und Francis Irving. Travel-Time Maps, 2007. URL
    % http://www.mysociety.org/2007/more-travel-maps/. Letzter Zugriff: 31.1.2010.

    %%%%%%%%%%%%%%%%%%%%%%%%%%%%%%%%%%%%%%%%%%%%%%%%%
    % 2011 Birchler: Isochrone.ch
    % Birchler, Marco (2013): isochrone.ch. Rapperswill: Hochschule für Technik Rapperswill. Online verfügbar unter http://www.isochrone.ch/, zuletzt geprüft am 27.12.2013.
    % https://www.ifs.hsr.ch/uploads/tx_icscrm/I_M.Birchler_H_2012.pdf
    % http://www.isochrone.ch/

    %%%%%%%%%%%%%%%%%%%%%%%%%%%%%%%%%%%%%%%%%%%%%%%%%
    % 2012 Meertens: Time Maps
    % http://www.timemaps.nl/
    % http://app.timemaps.nl/map vincent meertens
    % vincent meertens: Travel time maps of urban areas in the Netherlands
    % http://www.envplan.com/epa/fulltext/a44/a4590.pdf

    %%%%%%%%%%%%%%%%%%%%%%%%%%%%%%%%%%%%%%%%%%%%%%%%%
    % 2012 Soma: TripTropNYC
    % Soma, Jonathan (2012): TripTropNYC
    % http://www.triptropnyc.com/

    %%%%%%%%%%%%%%%%%%%%%%%%%%%%%%%%%%%%%%%%%%%%%%%%%
    % 2012 Wehrmeyer: Mapnificent
    % http://www.mapnificent.net/berlin/ stefan wehrmeyer
    % https://stefanwehrmeyer.com/projects/radnificent/ stefan wehrmeyer
    % https://stefanwehrmeyer.com/blog/2012/01/10/radnificent/ stefan wehrmeyer
    %
    % Wehrmeyer, Stefan (2013): Mapnificient. Bonn. Online verfügbar unter http://www.mapnificent.net/, zuletzt geprüft am 27.12.2013.
    %
    % @misc{wehrmeyer2012mapnificent,
    %   title={Mapnificent-Dynamic Public Transport Travel Time Maps},
    %   author={Wehrmeyer, S},
    %   year={2012},
    %   publisher={Retrieved}
    % }

    %%%%%%%%%%%%%%%%%%%%%%%%%%%%%%%%%%%%%%%%%%%%%%%%%
    % 2013 Melendez et al.: Transit Time NYC
    % http://project.wnyc.org/transit-time/
    % WNYC
    % Transit Time NYC
    % http://project.wnyc.org/transit-time/
    % Steven Melendez, John Keefe and Louise Ma / WNYC Data News Team. Follow us @datanews, email us here.

    %%%%%%%%%%%%%%%%%%%%%%%%%%%%%%%%%%%%%%%%%%%%%%%%%
    % 2014 Gortana et al.: Isoscope
    % http://isoscope.fh-potsdam.de/ marian dörk ?
    % http://isoscope.martinvonlupin.de/
    % http://www.flaviogortana.com/good/at/stuff/like/isoscope
    % https://uclab.fh-potsdam.de/projects/isoscope/
    %
    % Isoscope Mapping the time-varying extent of urban mobility
    %
    % A project 2014 by:
    % Flavio Gortana, Sebastian Kaim and Martin von Lupin supervised by Till Nagel during the class "Urbane Ebenen: Mobilität" at the University of Applied Sciences Potsdam
    % Gortana et al. (2014): Isoscope - Visualizing temporal mobility variance with isochrone maps

    %%%%%%%%%%%%%%%%%%%%%%%%%%%%%%%%%%%%%%%%%%%%%%%%%
    % 2015 iGeolise: TravelTimePlatform
    % http://www.igeolise.com/
    % http://www.traveltimeplatform.com/
    % https://data.gov.uk/apps/traveltime-platform
    % http://www.igeolise.com/2016/06/isochrone-map-driving-travel-time-geodistance/

  \section{Geographic visualizations of transportation networks}
    \label{sec:overv:geovs}

  \section{Web-based mapping components and services}
    \label{sec:overv:webmp}

  \section{Web-based rendering technologies}
    \label{sec:overv:webrn}



%%%%%%%%\chapter{Overview and related work}
%%%%%%%%  \label{chap:overv}

  %  Related work comprises basically accessibility map visualization, web-based
  %  visualization frameworks, and the rendering of transportation networks using
  %  GPUs. Glander et al. present an accessibility map visualization technique
  %  with a focus on polygon-based approaches~\cite{Glander2010}, and raster-based
  %  distance transforms~\cite{Mueller2010} which both lack precision in display
  %  offered by our approach. In~\cite{Yin2015}, a web-based system for visualization
  %  of multi-modal accessibility for multiple
  %  land-uses is presented. However, the visualization technique does not focus on the
  %  specifics of transportation network representations. Altmaier et al. (2003) was
  %  among the first to outline issues in web-based geovisualization
  %  applications~\cite{Altmaier2003}. In~\cite{Brabec2007}, challenges, requirements,
  %  and concepts of client-based browsing of spatial data on the World Wide Web are
  %  discussed. The presented system is based on a Java-Applet and does not exploit
  %  modern web technologies for rendering complex spatial data. Vaaraniemi et al. (2011)
  %  as well as Trapp et al. (2015) develop and evaluate approaches for transport network
  %  visualization utilizing modern graphics hardware \cite{Vaaraniemi2011,Trapp2015}.
  %  However, the presented rendering techniques can currently not be implemented
  %  using technologies for browser-based rendering and do no cover specifics of
  %  data representation and formats.\par

  % one way of communicating mobility information such as travel times are two dimensional maps
  % accessibility and mobility becomes a central topic in a growing number of fields
  % there are strong demands for corresponding web mapping components
  % based on massive geodata sets, such as osm

%%%%%%%%  \section{Accessibility analytics and mapping techniques}
%%%%%%%%    \label{sec:overv:accss}

%    A general motivation on web-based accessibility map applications in the context of geospatial analytics offers Hollburg et al. (2012) \cite{hollburghier}.\par

%    Glander et al. (2010) researches on visualization of accessibility maps with focus on polygon-based approaches \cite{Glander2010}.\par

    % isochronic passage chart (1881)
    % shortest path problem classes
    % Single pair shortest path (A to B)
    % Single source shortest path (A to N)
    % All pairs shortest path (N to M)

%%%%%%%%  \section{Geographic visualizations of transportation networks}
%%%%%%%%    \label{sec:overv:geovs}

    %Vaaraniemi et al. (2011) as well as Trapp et al. (2015) develop and evaluate solutions for transport network visualization utilizing modern computer graphics \cite{Vaaraniemi2011}\cite{Trapp2015}.\par

%%%%%%%%  \section{Web-based mapping components and services}
%%%%%%%%    \label{sec:overv:webmp}

    %Altmaier et al. (2003) was among the first to outline issues in web-based geovisualization applications \cite{Altmaier2003}. Klimke et al. (2011) proposed a camera path specification for geovisualization services on the web and mobile devices \cite{klimke2013service}.\par

%%%%%%%%  \section{Web-based rendering technologies}
%%%%%%%%    \label{sec:overv:webrn}

    %Both Coughlin (2013) and Trevett (2013) deliver outstanding motivations on why we need a standardized data format close to hardware devices for 3D applications on the web \cite{Coughlin2014}\cite{Trevett2012}.\par

    % raster, java, webgl
