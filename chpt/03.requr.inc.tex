%!TEX root = ../schoedon.tex

%%%%%%%%%%%%%%%%%%%%%%%%%%%%%%%%%%%%%%%%%%%%%%%%%%%%%%%%%%%%%%%%%%%%%%%%%%%%%%%%
\cleardoublepage              %%% REQUIREMENTS                               %%%
\chapter{Requirements for this approach}
  \label{chap:requr}
  \section{Network-based visualization}
    \label{sec:requr:netwr}
  \section{Compact data representation}
    \label{sec:requr:data}
  \section{Client-side mapping and rendering}
    \label{sec:requr:clien}
    % user foo interaction, dynamic bar
  \section{Web-based provision}
    \label{sec:requr:websd}
    This project focuses on the JavaScript webmapping framework \textit{Leaflet-JS} [??]. Better alternatives with a more advanced WebGL-integration are available (e.g. \textit{OpenLayers 3}) but a high compatibility with the existing \textit{Route360-JS} API developed by Motion Intelligence GmbH [??] is a requirement for this project. Therefore, Leaflet will be used.\par
    There is no native support for WebGL in Leaflet, yet\footnote{While writing this paper, a spanish software engineer, Iván Sánchez Ortega, started working on a Leaflet.GL plugin architecture [??] which was not taken into consideration as it is still experimental and was not available when this project implementation was started.}. Along with this implementation, a handful of Leaflet plugins will be developed which support a future WebGL integration [??][??] (see section {sec:tile}).\par
    % compare expert knowledge w/ GIS systems
%%%%%%%%%%%%%%%%%%%%%%%%%%%%%%%%%%%%%%%%%%%%%%%%%%%%%%%%%%%%%%%%%%%%%%%%%%%%%%%%
