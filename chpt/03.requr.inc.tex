%!TEX root = ../schoedon.tex

%%%%%%%%%%%%%%%%%%%%%%%%%%%%%%%%%%%%%%%%%%%%%%%%%%%%%%%%%%%%%%%%%%%%%%%%%%%%%%%%
\cleardoublepage              %%% REQUIREMENTS                               %%%
\chapter{Requirements for this approach}
  \label{chap:requr}

  One way of communicating mobility information such as travel times are
  two-dimensional maps. Accessibility and mobility becomes a central topic in a
  growing number of fields. There are strong demands for corresponding web
  mapping components based on massive geo-data sets, such as \acrshort{osm}.\par

  An interactive visualization technique for web-based accessibility maps should
  adhere to the following requirements and challenges:\par

  \begin{enumerate}[\label=({R}1)]
    \item \label{enu:requr:r1} It should support a web-based,
      hardware-accelerated implementation using WebGL \cite{Parisi2012}.
    \item \label{enu:requr:r2} It should utilize a standardized and compact
      data representation that allows for decoupling network geometry from
      temporal data to reduce data transmission and updates.
    \item \label{enu:requr:r3} It should enable interactive client-side
      filtering, mapping, and rendering for visual feedback.
    \item \label{enu:requr:r4} It should map the travel time information
      directly onto the transportation network for highly detailed
      visualizations.
  \end{enumerate}

  Requirement R\ref{enu:requr:r2}.

  \section{Web-based provision}
    \label{sec:requr:websd}
    This project focuses on the JavaScript webmapping framework \textit{Leaflet-JS} [??]. Better alternatives with a more advanced WebGL-integration are available (e.g. \textit{OpenLayers 3}) but a high compatibility with the existing \textit{Route360-JS} API developed by Motion Intelligence GmbH [??] is a requirement for this project. Therefore, Leaflet will be used.\par
    There is no native support for WebGL in Leaflet, yet\footnote{While writing this paper, a spanish software engineer, Iván Sánchez Ortega, started working on a Leaflet.GL plugin architecture [??] which was not taken into consideration as it is still experimental and was not available when this project implementation was started.}. Along with this implementation, a handful of Leaflet plugins will be developed which support a future WebGL integration [??][??] (see section {sec:tile}).\par
    % compare expert knowledge w/ GIS systems

  \section{Compact data representation}
    \label{sec:requr:data}

  \section{Client-side mapping and rendering}
    \label{sec:requr:clien}
    % user foo interaction, dynamic bar

  \section{Network-based visualization}
    \label{sec:requr:netwr}

%%%%%%%%%%%%%%%%%%%%%%%%%%%%%%%%%%%%%%%%%%%%%%%%%%%%%%%%%%%%%%%%%%%%%%%%%%%%%%%%
