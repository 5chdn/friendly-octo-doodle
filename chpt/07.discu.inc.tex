%!TEX root = ../schoedon.tex

%%%%%%%%%%%%%%%%%%%%%%%%%%%%%%%%%%%%%%%%%%%%%%%%%%%%%%%%%%%%%%%%%%%%%%%%%%%%%%%%
\cleardoublepage              %%% DISCUSSION                                 %%%
\chapter{Discussion of the results}
  % highlight positive stuff here
  %%
  %% Test Results & Discussion of results
  %%
  Figure~{WD:Fig:Performance} shows an overview of the performance comparison
  results (more precise results are presented in the supplemental material).
  These indicate superior run-time performance of the glTF approach over
  the native line rendering and GeoJSON implementation for mapping and rendering.
  Further, they show an increased scalability of the glTF approach w.r.t. the
  geometric complexity of the network and significant differences in mapping
  and WebGL 1.0 rendering performance for the different web browsers tested.\par
%%    \begin{table}[tb]
%%    %\beforeTable
%%    \centering
%%    \begin{tabular}{|l|r|r|r|r|r|}
%%      \hline
%%        & \multicolumn{5}{c|}{\textbf{Travel time in minutes}} \\\hline
%%       \textbf{Geometry} & 1  & 2   &     4 &      8 & 10\\\hline
%%      Vertices           & 36 & 162 & 1,901 & 14,874 & 25,821 \\
%%      Edges              & 20 & 88  & 1,006 &  7,950 & 13,829 \\
%%      \hline
%%    \end{tabular}
%%    \caption{Varying geometric complexity of the test data set.}
%%    \label{WD:Tab:TestData}
%%    \afterFloat
%%    \end{table}
  \section{Limitations of the presented approach}
  \section{Further possible application areas}
  \section{Recommendations for future work}
    Based on these results, future work focuses on developing a glTF processing
    back-end with separated tiling and routing server logic. The presented approach
    supports the decoupling of network geometry and accessibility data, which further
    reduces the amount of travel data transmission during updates. Further, a complete
    working client/server infrastructure is evaluated regarding it's overall
    performances to compare the results with traditional raster or vector approaches.\par

%%%%%%%%%%%%%%%%%%%%%%%%%%%%%%%%%%%%%%%%%%%%%%%%%%%%%%%%%%%%%%%%%%%%%%%%%%%%%%%%
