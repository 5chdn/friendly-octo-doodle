%!TEX root = ../schoedon.tex

%%%%%%%%%%%%%%%%%%%%%%%%%%%%%%%%%%%%%%%%%%%%%%%%%%%%%%%%%%%%%%%%%%%%%%%%%%%%%%%%
\cleardoublepage              %%% OUTLOOK                                    %%%
\chapter{Conclusions}
  A possible solution for a new approach of efficiently rendering and styling interactive web maps was demonstrated with the application focus on distance maps. Possible limitations were discovered and countermeasures developed.\par
  Geometry tiles utilizing preprocessed glTF data seem to be a promising way to render maps on both desktop and mobile devices. Future work should focus on developing a working glTF processing backend with tiling and routing server. On top of this, a fully working client/server infrastructure should be evaluated regarding it's performances. Results should be compared with classical raster or vector solutions.\par
  % focus on map applications composed of complex vector geometries
  % efficient mapping and rendering using gltf and webgl
  % superior run time performance over existing approaches
  % provides key approach for interactive mapping applications
  % provides highly detailed network visualization
  % scales to more complex transportation and street networks
    This paper presents a new approach for efficient rendering and mapping of
    interactive web-based accessibility-maps. The presented tiling approach based
    on glTF file format reduces the amount of data processing operations required
    by the client, increases run-time performance for efficient rendering, and
    simultaneously reduces the amount of transmitted data for visualization. With
    respect to rendering, the work compares this technique to two alternatives,
    showing a significant performance increase for massive data sets. The presented
    results enable the development of new interactive techniques for web-based
    transportation network visualization systems and tools.\par

%%%%%%%%%%%%%%%%%%%%%%%%%%%%%%%%%%%%%%%%%%%%%%%%%%%%%%%%%%%%%%%%%%%%%%%%%%%%%%%%
